\section{Задание №2(Вариант 51)}\label{02-lab}

\subsection{Условие}

Найти решение задачи линейного программирования симплекс-методом для целевой функции $F(x_1, x_2) = 3x_1 + x_2$.

\begin{align}
    F(x_1, x_2) = 3x_1 + x_2 \to \max\ \text{при условии} &
    \begin{cases}
        2x_1 + x_2 \leq 2\\
        -x_1 + 3x_2 \leq 3\\
        x_1 \geq 0\\
        x_2 \geq 0
    \end{cases}\tag{А}\label{02-lab-condition-a}\\
    F(x_1, x_2) = 3x_1 + x_2 \to \max(\min)\ \text{при условии} &
    \begin{cases}
        x_1 - 3x_2 \leq 2 \\
        x_1 + x_2 \geq 10 \\
        x_1 \geq 0 \\
        x_2 \geq 0
    \end{cases}\tag{Б}\label{02-lab-condition-b}\\
    F(x_1, x_2) = 3x_1 + x_2 \to \max\ \text{при условии} &
    \begin{cases}
        -x_1 + x_2 \geq 11\\
        x_1 - 4x_2 \geq 8\\
        x_1 \geq 0\\
        x_2 \geq 0
    \end{cases}\tag{В}
    \label{02-lab-condition-c}
\end{align}

\subsection{Решение}\label{02-lab-solution}
\subsubsection{Пункт \ref{02-lab-condition-a}}\label{02-lab-a}

Поставим задачу: наша компания продает товары А и Б. Количество продаж каждого товара - $x_1$ и $x_2$ соответственно.
Прибыль компании, $F(x_1, x_2) = 3x_1 + x_2$, нужно максимизировать. При этом на изготовление каждого товара мы тратим ресурсы U и V.
У нас есть ограничения на наличие ресурсов на складе. Пусть $x_3, x_4$ --- остаток ресурса U и V на складе соответственно.
Интерпретируем задачу математически:

\begin{align*}
    \begin{cases}
        F = 3x_1 + x_2\to max\\
        2x_1 + x_2 \leq 2\\
        -x_1 + 3 x_2 \leq 3\\
        x_1 \geq 0, x_2 \geq 0
    \end{cases} \longrightarrow
    \hspace{1.5cm}
    \begin{cases}
        F - 3x_1 - x_2 = 0\\
        2x_1 + x_2 + x_3 = 2\\
        -x_1 + 3x_2 + x_4 = 3\\
        x_{i} \geq 0, i\in\overline{1,4}
    \end{cases}
\end{align*}

Итак, нам нужно максимизировать $F = 3x_1 + x_2$. Используем для этого симплекс-метод. Он предполагает последовательную максимизацию
функции. 

Будем выходить из начальной точки: $x_1 = x_2 = x_3 = x_4 = 0$.
Выберем переменную: $x_1$ или $x_2$ -- которую выгодней сделать максимально возможной при наших условиях.
Из вида функции видно, что увеличение $x_1 \geq 0$ в большей степени увеличивает значение $F$, чем $x_2$.
Поэтому стараемся максимизировать $x_1$ максимально возможно при данных условиях, при этом оставляя $x_2 = 0$. 
Но насколько мы можем увеличить $x_1$, при этом сохраняя $x_2 = 0$ постоянным?

\[
    \begin{cases}
        2x_1 + x_2 + x_3 = 2\\
        -x_1 + 3x_2 + x_4 = 3\\
        x_{i} \geq 0, i\in\overline{1,4}
    \end{cases} 
    \Leftrightarrow
    \begin{cases}
        x_1 + \dfrac{1}{2}x_2 + \dfrac{1}{2}x_3 = 1\\ 
        -x_1 + 3x_2 + x_4 = 3\\
        x_{i} \geq 0, i\in\overline{1,4}
    \end{cases}
\]

Из первого уравнения системы видим $x_{1, max} = 1$, а из второго уравнения видно, что можно увеличивать $x_1$, не ограничиваясь.
Так как у нас система, то $x_{1, max} = 1$ --- меньшее неотрицательное.

Увеличим $x_1$ до $x_{1, max} = 1$. Теперь мы не можем увеличивать $x_1$, потому что достигнут лимит по условиям.
В таком случае зафиксируем $x_1 = 1$ и продолжим максимизировать $F$. При этом хотелось бы выразить $F$ через другие переменные,
еще можно увеличить. Сделаем так, чтобы $x_1$ пропал из всех уравнений, кроме одного. Сложим или вычтем уравнения таким образом, чтобы это получить:


\[
    \begin{cases}
        F = 3x_1 + x_2\\
        x_1 + \dfrac{1}{2}x_2 + \dfrac{1}{2}x_3 = 1\\ 
        -x_1 + 3x_2 + x_4 = 3\\
        x_{i} \geq 0, i\in\overline{1,4}
    \end{cases}
    \Leftrightarrow
    \begin{cases}
        3x_1 + x_2 - F = 0\\
        x_1 + \dfrac{1}{2}x_2 + \dfrac{1}{2}x_3 = 1\\
        -x_1 + 3x_2 + x_4 = 3\\
        x_{i} \geq 0, i\in\overline{1,4}
    \end{cases}
    \Leftrightarrow
    \begin{cases}
        -x_2 - \dfrac{3}{2}x_3 - F = -3\\
        x_1 + \dfrac{1}{2}x_2 + \dfrac{1}{2}x_3 = 1\\
        \dfrac{7}{2}x_2 + \dfrac{1}{2}x_3 + x_4 = 4\\
        x_{i} \geq 0, i\in\overline{1,4}
    \end{cases}
\]

Вот во что превратилось выражение для $F$: $F = -x_2 - \dfrac{3}{2}x_3 + 3$. Анализируя это выражение, приходим к выводу, что
мы достигли максимума $F$, так как, какую бы переменную не увеличивай, $F$ уменьшится. Мы решили задачу нахождения максимума, осталось только дать ответ.
Итак, мы зафиксировали $x_1 = 1$, при этом, чтобы удовлетворить второе условие, $x_4 = 4$. Из первого условия $x_3 = 0$, а $x_2 = 0$, так как мы его намеренно не меняли.

Подставим данные значения в получившуюся целевую функцию: 
$$F = -x_2 - \dfrac{3}{2}x_3 + 3 = -0 - \dfrac{3}{2} * 0 + 3 = 3 = \max F.\\ \argmax F = (x_1, x_2) = (1, 0)$$

Данные рассуждения повторились бы, если бы целевая функция могла увеличиться еще. И наш цикл повторился бы еще раз.

Данный способ - алгебраический способ решения задачи линейного программирования. Можно проиллюстрировать работу этого алгоритма на графике \ref{fig:01-lab-01-graphic}:

Точка старта - начало координат. $x_1$ было увеличивать выгоднее, поэтому мы пошли по оси $Ox$ вправо, пока не достигли
граничного значения в точке $3:(1, 0)$. Нам повезло, мы попали в точку максимума на первом цикле. В общем случае делается обход границы области определения.

На практике алгоритм данного метода можно описать множеством таблиц --- симплекс-таблиц. Каждый цикл --- это переход между симплекс-таблицами.

Построим множество симплекс-таблиц и с помощью них решим ту же задачу оптимизации:

\begin{table}[H]
    \centering
    \begin{tabular}{|c|>{\columncolor[HTML]{98FB98}}c|c|c|c|c|c|}
        \hline
        Базис & $x_1$ & $x_2$ & $x_3$ & $x_4$ & $b_i$ & $\dfrac{b_i}{\text{решающий столбец}}$ \\
        \hline
        \rowcolor[HTML]{E0FFFF}
        $x_3$ & \cellcolor[HTML]{BDFDCC} 2 & 1 & 1 & 0 & 2 & 1 \leftarrow min\\
        \hline
        $x_4$ & {-1} & 3 & 0 & 1 & 3 & -3 < 0\\
        \hline
        F & -3 \leftarrow min & -1 & 0 & 0 & 0 & \\
        \hline
    \end{tabular}
    \caption{}
    \label{02-lab-01-table}
\end{table}

В данном случае нашу функцию можно представить как $F = 3x_1 + x_2$. Теперь становится ясно, за что отвечает отрицательный минимум.
Выбрали столбец. Теперь в последнем столбце считаем максимальное увеличение $x_1$. Как уже было показано ранее $x_{1, max} = 1$.
Сначала мы выбрали элемент, который станет базисным, а теперь мы выбрали элемент, который станет свободным, отдавая место $x_1$ -- это $x_3$.
Делаем то же самое, что и в системе уравнений: складываем, вычитаем, домножаем строчки. Причем, как в линейной алгебре, 
можно создавать любую линейную комбинацию, главное учитывать особенности данной таблицы --- множители перед $F$ и последний столбец, 
который не меняется от домножения:

\begin{table}[H]
    \centering
    \begin{tabular}{|c|c|c|c|c|c|}
        \hline
        \cellcolor[HTML]{BDFDCC}F & -3 & -1 & 0 & 0 & 0\\
        \hline
        \cellcolor[HTML]{BDFDCC}2F & -6 & -2 & 0 & 0 & 0\\
        \hline
        \cellcolor[HTML]{BDFDCC}-F & 3 & 1 & 0 & 0 & 0\\
        \hline
    \end{tabular}
    \caption{}
    \label{02-lab-02-table}
\end{table}

Теперь сделаем $x_1$ базисным:

\begin{table}[H]
    \centering
    \begin{tabular}{|p{0.75cm}|p{1cm}|>{\columncolor[HTML]{98FB98}}c|c|c|c|c|c|}
        \hline
        &Базис & $x_1$ & $x_2$ & $x_3$ & $x_4$ & $b_i$ & $\dfrac{b_i}{\text{решающий столбец}}$ \\
        \hline
        \rowcolor[HTML]{E0FFFF}
        &$x_3$ & \cellcolor[HTML]{BDFDCC}2 & 1 & 1 & 0 & 2 & 1 \leftarrow min\\
        \hline
        \textbf{(*2)}&$x_4$ & {-2} & 6 & 0 & 2 & 6 & -3 < 0\\
        \hline
        \textbf{(*2)}& 2F & -6 \leftarrow min & -2 & 0 & 0 & 0 & \\
        \hline
    \end{tabular}
    \caption{}
    \label{02-lab-03-table}
\end{table}

\begin{table}[H]
    \centering
    \begin{tabular}{|c|c|c|c|c|c|c|c|}
        \hline
        &Базис & $x_1$ & $x_2$ & $x_3$ & $x_4$ & $b_i$ & $\dfrac{b_i}{\text{решающий столбец}}$ \\
        \hline
        &$x_1$ & 2 & 1 & 1 & 0 & 2 &\\
        \hline
        &$x_4$ & 0 & 7 & 1 & 2 & 8 &\\
        \hline
        &2F & 0 & 1 & 3 & 0 & 6 & Все > 0\\
        \hline
    \end{tabular}
    \caption{}
    \label{02-lab-04-table}
\end{table}

Таким образом выражение для нашей функции превратилось в $2F = -x_2 - 3x_3 + 6$ -- сравните с ответом выше.
Отсутствие отрицательных элементов в строчке при F --- окончание алгоритма симплекс-метода.

Аргументы максимума находятся как $x_i = \begin{cases}
    0, &x_i \text{-- не базисный}\\
    \frac{b_i}{x_{\text{базисный}, i}} &x_i \text{-- базисный}
\end{cases}$, далее подставляются в исходную функцию. Значение после подстановки 
и в правой нижней ячейке при учете множителя перед $F$ должны совпасть. Причем на любой итерации, не обязательно на конечной.


Как видно из таблицы, базисные элементы оптимального метода равны $x_1 = 1, x_4 = 4$, а $x_2 = 0, x_3 = 0$.

$\max F(x_1, x_2) = 3$, $\argmax F(x_1, x_2) = (1, 0)$, т.к. $x_1 = 1, x_2 = 0$ из строки выше.

\textbf{Ответ:} $\max F(x_1, x_2) = 3$, $\argmax F(x_1, x_2) = (1, 0)$ \label{02-lab-a-answer}

Далее решение будет идти без подробностей.

\subsubsection{Пункт \ref{02-lab-condition-b}} \label{02-lab-b}

Приведем задачу к каноническому виду(для $max$), путём введения базиса $x_3, x_4$:

\begin{align*}
    \begin{cases}
        F = 3x_1 + x_2\to max\\
        x_1 - 3x_2 \leq 3\\
        x_1 + x_2 \geq 10\\
        x_1 \geq 0, x_2 \geq 0
    \end{cases} \longrightarrow
    \hspace{1.5cm}
    \begin{cases}
        F - 3x_1 - x_2 = 0\\
        x_1 - 3x_2 + x_3 = 3\\
        -x_1 - x_2 + x_4 = -10\\
        x_{i} \geq 0, i\in\overline{1,4}
    \end{cases}
\end{align*}

\paragraph{$F(x_1, x_2) \to \max$}

Построим симплекс-таблицу и с помощью неё решим задачу оптимизации:

\begin{table}[H]
    \centering
    \begin{tabular}{|c|c|>{\columncolor{mycolumncolor}}c|c|c|c|c|c|}
    \hline
         & Базис &  $x_1$ & $x_2$ & $x_3$ & $x_4$ & $b_i$ & $\frac{b_i}{\text{разрешающий столбец}}$ \\ \hline
         \myrowcolor
        ~ & $x_3$ & \mycellcolor1 & -3 & 1 & 0 & 3 & 3 \leftarrow min \\ \hline
        ~ & $x_4$ & -1 & -1 & 0 & 1 & -10 & 10 \\ \hline
        ~ & F & -3 \leftarrow min & -1 & 0 & 0 & 0 & ~ \\ \hline
    \end{tabular}
    \caption{}
    \label{02-lab-05-table}
\end{table}

\begin{table}[H]
    \centering
    \begin{tabular}{|c|c|c|>{\columncolor{mycolumncolor}}c|c|c|c|c|}
    \hline
         & Базис & $x_1$ & $x_2$ & $x_3$ & $x_4$ & $b_i$ & $\frac{b_i}{\text{разрешающий столбец}}$ \\ \hline
         \myrowcolor
        ~ & $x_1$ & 1 & -3 \mycellcolor &  1 & 0 & 3 & -1 < 0 \\ \hline
        ~ & $x_4$ & 0 & -4 & 1 & 1 & -7 & $\frac{7}{4}$ \leftarrow min\\\hline
        ~ & F & 0 & -10 \leftarrow min & 3  & 0 & 9 & ~ \\ \hline
    \end{tabular}
    \caption{}
    \label{02-lab-06-table}
\end{table}

\begin{table}[H]
    \centering
    \begin{tabular}{|c|c|c|>{\columncolor{mycolumncolor}}c|c|c|c|c|}
    \hline
         & Базис & $x_1$ & $x_2$ & $x_3$ & $x_4$ & $b_i$ & $\frac{b_i}{\text{разрешающий столбец}}$ \\ \hline
        \textbf{*4} & $x_1$ & 4 & -12 & 4 & 0 & 12 & -1 < 0 \\ \hline
        \myrowcolor
        \textbf{*(-1)} & $x_4$ & 0 & 4\mycellcolor & -1 & -1 & 7 & $\frac{7}{4}$ \leftarrow min \\ \hline
        \textbf{*2} & 2F & 0 & -20 \leftarrow min & 6 & 0 & 18 & ~ \\ \hline
    \end{tabular}
    \caption{}
    \label{02-lab-07-table}
\end{table}

\begin{table}[H]
    \centering
    \begin{tabular}{|c|c|c|c|c|>{\columncolor{mycolumncolor}}c|c|c|}
    \hline
         & Базис & $x_1$ & $x_2$ & $x_3$ & $x_4$ & $b_i$ & $\frac{b_i}{\text{разрешающий столбец}}$ \\ \hline
         & $x_1$ & 4 & 0 & 1 & -3 & 33 & -11 < 0 \\ \hline
         & $x_2$ & 0 & 4 & -1 & -1 & 7 & - 7 < 0 \\ \hline
         & 2F & 0 & 0 & 1 & -5 \leftarrow min & 53 & ~ \\ \hline
    \end{tabular}
    \caption{}
    \label{02-lab-08-table}
\end{table}

Так как мы пришли к выражению $2F = -x_3 + 5x_4 + 53$, 
то мы должны максимизировать $x_4$, однако мы можем максимизировать его бесконечно,
что означает неограниченность области определения.

\textbf{Ответ:} $\max F(x_1, x_2) \notin \R, \argmax F(x_1, x_2) \notin \R^2$ \label{02-lab-b-max-answer}

\paragraph{$F(x_1, x_2) \to \min$}

Пусть $G = -F(x_1, x_2)$, тогда $F \to \min \Leftrightarrow G \to \max$.

Таким образом, мы решаем задачу:

\begin{align*}
    \begin{cases}
        G = -3x_1 - x_2\to max\\
        x_1 - 3x_2 \leq 3\\
        x_1 + x_2 \geq 10\\
        x_1 \geq 0, x_2 \geq 0
    \end{cases} \longrightarrow
    \hspace{1.5cm}
    \begin{cases}
        G + 3x_1 + x_2 = 0\\
        x_1 - 3x_2 + x_3 = 3\\
        -x_1 - x_2 + x_4 = -10\\
        x_{i} \geq 0, i\in\overline{1,4}
    \end{cases}
\end{align*}

В симплекс-методе мы начинаем с точки (0, 0), но очевидно что она не входит в область определения,
поэтому мы сделаем первый шаг, который уменьшит $G$, но дойдет до точки из области определения.

\begin{table}[H]
    \centering
    \begin{tabular}{|c|c|c|>{\columncolor{mycolumncolor}}c|c|c|c|c|}
    \hline
         & Базис & $x_1$ & $x_2$ & $x_3$ & $x_4$ & $b_i$ & $\frac{b_i}{\text{разрешающий столбец}}$ \\ \hline
         & $x_3$ & 1 & -3 & 1 & 0 & 3 & -1 < 0 \\ \hline
         \myrowcolor
         & $x_4$ & -1 & \mycellcolor-1 & 0 & 1 & -10 & 10 \leftarrow min \\ \hline
         & G & 3 & 1 \leftarrow min & 0 & 0 & 0 & ~ \\ \hline
    \end{tabular}
    \caption{}
    \label{02-lab-09-table}
\end{table}

\begin{table}[H]
    \centering
    \begin{tabular}{|c|c|c|>{\columncolor{mycolumncolor}}c|c|c|c|c|}
    \hline
         & Базис & $x_1$ & $x_2$ & $x_3$ & $x_4$ & $b_i$ & $\frac{b_i}{\text{разрешающий столбец}}$ \\ \hline
         & $x_3$ & 1 & -3 & 1 & 0 & 3 & -1 < 0 \\ \hline
         \myrowcolor
        \textbf{*(-1)} & $x_4$ & 1 & \mycellcolor1 & 0 & -1 & 10 & 10 \leftarrow min \\ \hline
         & G & 3 & 1 \leftarrow min & 0 & 0 & 0 & ~ \\ \hline
    \end{tabular}
    \caption{}
    \label{02-lab-10-table}
\end{table}

\begin{table}[H]
    \centering
    \begin{tabular}{|c|c|c|c|c|c|c|c|}
    \hline
         & Базис & $x_1$ & $x_2$ & $x_3$ & $x_4$ & $b_i$ & $\frac{b_i}{\text{разрешающий столбец}}$ \\ \hline
         & $x_3$ & 4 & 0 & 1 & -3 & 33 &\\ \hline
         & $x_2$ & 1 & 1 & 0 & -1 & 10 & \\ \hline
         & G & 2 & 0 & 0 & 1 & -10 & Все > 0 \\ \hline
    \end{tabular}
    \caption{}
    \label{02-lab-11-table}
\end{table}

Мы сместились в точку $x_1 = 0, x_2 = 10, x_3 = 33, x_4 = 0$. Она входит в область определения. Мы пришли к следующему выражению:
$G = -2x_1 - x_4$. Мы привели $G \to \max$. А значит и $F = -G = 2x_1 + x_4 \to \min$.
Мы знаем, что $\argmin F = (0, 10)$, тогда $\min F = F(0, 10) = 3 * 0 + 1 * 10 = 10$, что подтверждает \hyperref[01-lab-02-graphic]{геометрический способ решения} данного задания.

\textbf{Ответ:} $\min F = 10, \argmin F = (0, 10)$\label{02-lab-b-min-answer}

\subsubsection{Пункт \ref{02-lab-condition-c}}\label{02-lab-c}

Приведем задачу к каноническому виду(для $max$), путём введения базиса $x_3, x_4$:

\[
    \begin{cases}
        F(x_1, x_2) = 3x_1 + x_2 \to \max\\
        -x_1 + x_2 \geq 11\\
        x_1 - 4x_2 \geq 8\\
        x_1 \geq 0\\
        x_2 \geq 0
    \end{cases}
    \longrightarrow
    \hspace{1.5cm}
    \begin{cases}
        F - 3x_1 - x_2 = 0\\
        -x_1 + x_2 - x_3 = 11\\
        x_1 - 4x_2 - x_4 = 8\\
        x_{i} \geq 0, i \in \overline{1, 4}
    \end{cases}
\]

Начинаем рассчитывать симплекс-таблицы:

\begin{table}[H]
    \centering
    \begin{tabular}{|c|c|>{\columncolor{mycolumncolor}}c|c|c|c|c|c|}
    \hline
         & Базис & $x_1$ & $x_2$ & $x_3$ & $x_4$ & $b_i$ & $\frac{b_i}{\text{разрешающий столбец}}$ \\ \hline
         & $x_3$ & -1 & 1 & 1 & 0 & 11 & -11 < 0 \\ \hline
         \myrowcolor
         & $x_4$ & \mycellcolor1 & -4 & 0 & 1 & 8 & 8 \leftarrow min \\ \hline
         & F & -3 \leftarrow min & -1 & 0 & 0 & 0 & ~ \\ \hline
    \end{tabular}
    \caption{}
    \label{02-lab-12-table}
\end{table}

\begin{table}[H]
    \centering
    \begin{tabular}{|c|c|c|>{\columncolor{mycolumncolor}}c|c|c|c|c|}
    \hline
         & Базис & $x_1$ & $x_2$ & $x_3$ & $x_4$ & $b_i$ & $\frac{b_i}{\text{разрешающий столбец}}$ \\ \hline
         & $x_3$ & 0 & -3 & 1 & 1 & 19 & $-\frac{19}{3}$ < 0 \\ \hline
         & $x_1$ & 1 & -4 & 0 & 1 & 8 & -2 < 0 \\ \hline
         & F & 0 & -13 \leftarrow min & 0 & 3 & 24 & ~ \\ \hline
    \end{tabular}
    \caption{}
    \label{02-lab-13-table}
\end{table}

Мы сместились в точку (8, 0), максимизируя целевую функцию $F$ по $x_1$.
Точка (8, 0), исходя из системы неравенств, задающую условие, не удовлетворяет ему. 
Точка (8, 0) не пренадлежит области определения. При этом мы достигли максимума $F$ по $x_1$ и двигаться дальше не можем,
потому что $F$ по $x_2$ дальше не максимизируется.
Отсюда вывод, что область определения функции пуста. Максимума функции, как и точки максимума не существует.

Эту ситуацию наглядно показывает \hyperref[01-lab-03-graphic]{график из задания №1}.

\textbf{Ответ:} $\nexists\max F(x_1, x_2), \nexists\argmax F(x_1, x_2).$ \label{02-lab-c-answer}

\newpage