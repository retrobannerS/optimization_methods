\section{Задание №5}\label{05-lab}

\subsection{Условие}\label{05-lab-condition}

Придумать транспортную задачу закрытого типа со следующими ограничениями:
\begin{enumerate}
    \item Размерность $5 \times 4$.
    \item Запасы и потребности в диапазоне от 10 до 200.
    \item Стоимость поставок от 1 до 9.
\end{enumerate}

Найти начальное опорное решение методом северо-западного угла и методом минимальных элементов.
Понравившийся метод использовать для решения задачи методом потенциалов.
Если в методе потенциалов сразу получилось оптимальное решение, то взять другую таблицу в качестве начального опорного решения.

\subsection{Постановка задачи}\label{05-lab-statement}

\begin{enumerate}
    \item Запасы на 5 складах: $\begin{pmatrix}
                  170 & 65 & 115 & 130 & 140
              \end{pmatrix}$
    \item Потребности 4 магазинов: $\begin{pmatrix}
                  195 & 190 & 135 & 100
              \end{pmatrix}$
    \item Стоимость поставок: $\begin{pmatrix}
                  7 & 4 & 8 & 8 \\
                  7 & 3 & 7 & 8 \\
                  5 & 4 & 8 & 8 \\
                  3 & 6 & 5 & 2 \\
                  8 & 6 & 2 & 5
              \end{pmatrix}$
    \item Целевая функция: $f = 7x_{11} + 4x_{12} + 8x_{13} + 8x_{14} + 7x_{21} + 3x_{22} + 7x_{23} + 8x_{24} + 5x_{31} + 4x_{32} + 8x_{33} + 8x_{34} + 3x_{41} + 6x_{42} + 5x_{43} + 2x_{44} + 8x_{51} + 6x_{52} + 2x_{53} + 5x_{54} \to \min$

    \item Ограничения:
          \[\begin{cases}
                  x_{11} + x_{12} + x_{13} + x_{14} = 170          \\
                  x_{21} + x_{22} + x_{23} + x_{24} = 65           \\
                  x_{31} + x_{32} + x_{33} + x_{34} = 115          \\
                  x_{41} + x_{42} + x_{43} + x_{44} = 130          \\
                  x_{51} + x_{52} + x_{53} + x_{54} = 140          \\
                  x_{11} + x_{21} + x_{31} + x_{41} + x_{51} = 195 \\
                  x_{12} + x_{22} + x_{32} + x_{42} + x_{52} = 190 \\
                  x_{13} + x_{23} + x_{33} + x_{43} + x_{53} = 135 \\
                  x_{14} + x_{24} + x_{34} + x_{44} + x_{54} = 100 \\
              \end{cases}\]
\end{enumerate}

\subsection{Решение}\label{05-lab-solution}

\subsubsection{Метод северо-западного угла}\label{05-lab-nw-corner}

Начнём с пустой таблицы, где в уголочках запишем стоимость поставок.

\begin{table}[H]
    \centering
    \begin{tabular}{|c|c|c|c|c|c|}
        \hline
        \backslashbox{Склад}{Магазин} & 1                & 2                & 3                & 4                & Запас              \\
        \hline
        1                             & \doublecell{}{7} & \doublecell{}{4} & \doublecell{}{8} & \doublecell{}{8} & 170                \\
        \hline
        2                             & \doublecell{}{7} & \doublecell{}{3} & \doublecell{}{7} & \doublecell{}{8} & 65                 \\
        \hline
        3                             & \doublecell{}{5} & \doublecell{}{4} & \doublecell{}{8} & \doublecell{}{8} & 115                \\
        \hline
        4                             & \doublecell{}{3} & \doublecell{}{6} & \doublecell{}{5} & \doublecell{}{2} & 130                \\
        \hline
        5                             & \doublecell{}{8} & \doublecell{}{6} & \doublecell{}{2} & \doublecell{}{5} & 140                \\
        \hline
        Потребность                   & 195              & 190              & 135              & 100              & \diagbox{620}{620} \\
        \hline
    \end{tabular}
\end{table}

Заполним таблицу по методу северо-западного угла. Начинаем с клетки $(1, 1)$,
заполняем её на $x_{11} = \min(170, 195) = 170$,
вычитаем из запаса на складе и из потребности магазина это значение,
проставляем '--' там, где образовался ноль.

\begin{table}[H]
    \centering
    \begin{tabular}{|c|c|c|c|c|c|}
        \hline
        \backslashbox{Склад}{Магазин} & 1                   & 2                   & 3                   & 4                   & Запас              \\
        \hline
        1                             & \doublecell{170}{7} & \doublecell{$-$}{4} & \doublecell{$-$}{8} & \doublecell{$-$}{8} & \cancel{170} 0     \\
        \hline
        2                             & \doublecell{}{7}    & \doublecell{}{3}    & \doublecell{}{7}    & \doublecell{}{8}    & 65                 \\
        \hline
        3                             & \doublecell{}{5}    & \doublecell{}{4}    & \doublecell{}{8}    & \doublecell{}{8}    & 115                \\
        \hline
        4                             & \doublecell{}{3}    & \doublecell{}{6}    & \doublecell{}{5}    & \doublecell{}{2}    & 130                \\
        \hline
        5                             & \doublecell{}{8}    & \doublecell{}{6}    & \doublecell{}{2}    & \doublecell{}{5}    & 140                \\
        \hline
        Потребность                   & \cancel{195} 25     & 190                 & 135                 & 100                 & \diagbox{450}{450} \\
        \hline
    \end{tabular}
\end{table}

Следующую итерацию начнём снова с левой верхней клетки, в данном случае $(2, 1)$,
заполним её на $x_{21} = \min(65, 25) = 25$,
вычтем из запаса на складе и из потребности магазина это значение,
проставим '--' там, где образовался ноль.

\begin{table}[H]
    \centering
    \begin{tabular}{|c|c|c|c|c|c|}
        \hline
        \backslashbox{Склад}{Магазин} & 1                          & 2                   & 3                   & 4                   & Запас              \\
        \hline
        1                             & \doublecell{170}{7}        & \doublecell{$-$}{4} & \doublecell{$-$}{8} & \doublecell{$-$}{8} & \cancel{170} 0     \\
        \hline
        2                             & \doublecell{25}{7}         & \doublecell{}{3}    & \doublecell{}{7}    & \doublecell{}{8}    & \cancel{65} 40     \\
        \hline
        3                             & \doublecell{$-$}{5}        & \doublecell{}{4}    & \doublecell{}{8}    & \doublecell{}{8}    & 115                \\
        \hline
        4                             & \doublecell{$-$}{3}        & \doublecell{}{6}    & \doublecell{}{5}    & \doublecell{}{2}    & 130                \\
        \hline
        5                             & \doublecell{$-$}{8}        & \doublecell{}{6}    & \doublecell{}{2}    & \doublecell{}{5}    & 140                \\
        \hline
        Потребность                   & \cancel{170} \cancel{25} 0 & 190                 & 135                 & 100                 & \diagbox{425}{425} \\
        \hline
    \end{tabular}
\end{table}

Следующая итерация:

\begin{table}[H]
    \centering
    \begin{tabular}{|c|c|c|c|c|c|}
        \hline
        \backslashbox{Склад}{Магазин} & 1                          & 2                   & 3                   & 4                   & Запас                     \\
        \hline
        1                             & \doublecell{170}{7}        & \doublecell{$-$}{4} & \doublecell{$-$}{8} & \doublecell{$-$}{8} & \cancel{170} 0            \\
        \hline
        2                             & \doublecell{25}{7}         & \doublecell{40}{3}  & \doublecell{$-$}{7} & \doublecell{$-$}{8} & \cancel{65} \cancel{40} 0 \\
        \hline
        3                             & \doublecell{$-$}{5}        & \doublecell{}{4}    & \doublecell{}{8}    & \doublecell{}{8}    & 115                       \\
        \hline
        4                             & \doublecell{$-$}{3}        & \doublecell{}{6}    & \doublecell{}{5}    & \doublecell{}{2}    & 130                       \\
        \hline
        5                             & \doublecell{$-$}{8}        & \doublecell{}{6}    & \doublecell{}{2}    & \doublecell{}{5}    & 140                       \\
        \hline
        Потребность                   & \cancel{170} \cancel{25} 0 & \cancel{190} 150    & 135                 & 100                 & \diagbox{390}{390}        \\
        \hline
    \end{tabular}
\end{table}

Делаем данные итерации, пока не заполним всю таблицу. Готовая таблица:

\begin{table}[H]
    \centering
    \begin{tabular}{|c|c|c|c|c|c|}
        \hline
        \backslashbox{Склад}{Магазин} & 1                   & 2                   & 3                   & 4                   & Запас          \\
        \hline
        1                             & \doublecell{170}{7} & \doublecell{$-$}{4} & \doublecell{$-$}{8} & \doublecell{$-$}{8} & 0              \\
        \hline
        2                             & \doublecell{25}{7}  & \doublecell{40}{3}  & \doublecell{$-$}{7} & \doublecell{$-$}{8} & 0              \\
        \hline
        3                             & \doublecell{$-$}{5} & \doublecell{115}{4} & \doublecell{$-$}{8} & \doublecell{$-$}{8} & 0              \\
        \hline
        4                             & \doublecell{$-$}{3} & \doublecell{35}{6}  & \doublecell{95}{5}  & \doublecell{$-$}{2} & 0              \\
        \hline
        5                             & \doublecell{$-$}{8} & \doublecell{$-$}{6} & \doublecell{40}{2}  & \doublecell{100}{5} & 0              \\
        \hline
        Потребность                   & 0                   & 0                   & 0                   & 0                   & \diagbox{0}{0} \\
        \hline
    \end{tabular}
\end{table}

Мы получили опорное решение, которое можно улучшить методом потенциалов.

\subsubsection{Метод минимальных элементов}\label{05-lab-min}

Начнём с пустой таблицы, где в уголочках запишем стоимость поставок.

\begin{table}[H]
    \centering
    \begin{tabular}{|c|c|c|c|c|c|}
        \hline
        \backslashbox{Склад}{Магазин} & 1                & 2                & 3                & 4                & Запас              \\
        \hline
        1                             & \doublecell{}{7} & \doublecell{}{4} & \doublecell{}{8} & \doublecell{}{8} & 170                \\
        \hline
        2                             & \doublecell{}{7} & \doublecell{}{3} & \doublecell{}{7} & \doublecell{}{8} & 65                 \\
        \hline
        3                             & \doublecell{}{5} & \doublecell{}{4} & \doublecell{}{8} & \doublecell{}{8} & 115                \\
        \hline
        4                             & \doublecell{}{3} & \doublecell{}{6} & \doublecell{}{5} & \doublecell{}{2} & 130                \\
        \hline
        5                             & \doublecell{}{8} & \doublecell{}{6} & \doublecell{}{2} & \doublecell{}{5} & 140                \\
        \hline
        Потребность                   & 195              & 190              & 135              & 100              & \diagbox{620}{620} \\
        \hline
    \end{tabular}
\end{table}

Смотрим на стоимости поставок и выбираем клетку, в которой стоимость минимальна. В данном случае это клетка $(4, 4)$,
заполняем её на $x_{44} = \min(130, 100) = 100$, вычитаем из запаса на складе и из потребности магазина это значение,
проставляем '--' там, где образовался ноль.

\begin{table}[H]
    \centering
    \begin{tabular}{|c|c|c|c|c|c|}
        \hline
        \backslashbox{Склад}{Магазин} & 1                & 2                & 3                & 4                   & Запас              \\
        \hline
        1                             & \doublecell{}{7} & \doublecell{}{4} & \doublecell{}{8} & \doublecell{$-$}{8} & 170                \\
        \hline
        2                             & \doublecell{}{7} & \doublecell{}{3} & \doublecell{}{7} & \doublecell{$-$}{8} & 65                 \\
        \hline
        3                             & \doublecell{}{5} & \doublecell{}{4} & \doublecell{}{8} & \doublecell{$-$}{8} & 115                \\
        \hline
        4                             & \doublecell{}{3} & \doublecell{}{6} & \doublecell{}{5} & \doublecell{100}{2} & \cancel{130} 30    \\
        \hline
        5                             & \doublecell{}{8} & \doublecell{}{6} & \doublecell{}{2} & \doublecell{$-$}{5} & 140                \\
        \hline
        Потребность                   & 195              & 190              & 135              & \cancel{100} 0      & \diagbox{520}{520} \\
        \hline
    \end{tabular}
\end{table}

Далее минимальная стоимость поставки в клетке $ (5, 3) $,
заполняем её на $ x_{53} = \min(140, 135) = 135 $,
вычитаем из запаса на складе и из потребности магазина это значение,
проставляем '--' там, где образовался ноль.

\begin{table}[H]
    \centering
    \begin{tabular}{|c|c|c|c|c|c|}
        \hline
        \backslashbox{Склад}{Магазин} & 1                & 2                & 3                   & 4                   & Запас              \\
        \hline
        1                             & \doublecell{}{7} & \doublecell{}{4} & \doublecell{$-$}{8} & \doublecell{$-$}{8} & 170                \\
        \hline
        2                             & \doublecell{}{7} & \doublecell{}{3} & \doublecell{$-$}{7} & \doublecell{$-$}{8} & 65                 \\
        \hline
        3                             & \doublecell{}{5} & \doublecell{}{4} & \doublecell{$-$}{8} & \doublecell{$-$}{8} & 115                \\
        \hline
        4                             & \doublecell{}{3} & \doublecell{}{6} & \doublecell{$-$}{5} & \doublecell{100}{2} & \cancel{130} 30    \\
        \hline
        5                             & \doublecell{}{8} & \doublecell{}{6} & \doublecell{135}{2} & \doublecell{$-$}{5} & \cancel{140} 5     \\
        \hline
        Потребность                   & 195              & 190              & \cancel{135} 0      & \cancel{100} 0      & \diagbox{385}{385} \\
        \hline
    \end{tabular}
\end{table}

Следующая итерация:

\begin{table}[H]
    \centering
    \begin{tabular}{|c|c|c|c|c|c|}
        \hline
        \backslashbox{Склад}{Магазин} & 1                   & 2                  & 3                   & 4                   & Запас              \\
        \hline
        1                             & \doublecell{}{7}    & \doublecell{}{4}   & \doublecell{$-$}{8} & \doublecell{$-$}{8} & 170                \\
        \hline
        2                             & \doublecell{$-$}{7} & \doublecell{65}{3} & \doublecell{$-$}{7} & \doublecell{$-$}{8} & \cancel{65} 0      \\
        \hline
        3                             & \doublecell{}{5}    & \doublecell{}{4}   & \doublecell{$-$}{8} & \doublecell{$-$}{8} & 115                \\
        \hline
        4                             & \doublecell{}{3}    & \doublecell{}{6}   & \doublecell{$-$}{5} & \doublecell{100}{2} & \cancel{130} 30    \\
        \hline
        5                             & \doublecell{}{8}    & \doublecell{}{6}   & \doublecell{135}{2} & \doublecell{$-$}{5} & \cancel{140} 5     \\
        \hline
        Потребность                   & 195                 & \cancel{190} 125   & \cancel{135} 0      & \cancel{100} 0      & \diagbox{320}{320} \\
        \hline
    \end{tabular}
\end{table}

И так далее пока не заполним всю таблицу. Готовая таблица:

\begin{table}[H]
    \centering
    \begin{tabular}{|c|c|c|c|c|c|}
        \hline
        \backslashbox{Склад}{Магазин} & 1                   & 2                   & 3                   & 4                   & Запас          \\
        \hline
        1                             & \doublecell{160}{7} & \doublecell{10}{4}  & \doublecell{$-$}{8} & \doublecell{$-$}{8} & 0              \\
        \hline
        2                             & \doublecell{$-$}{7} & \doublecell{65}{3}  & \doublecell{$-$}{7} & \doublecell{$-$}{8} & 0              \\
        \hline
        3                             & \doublecell{$-$}{5} & \doublecell{115}{4} & \doublecell{$-$}{8} & \doublecell{$-$}{8} & 0              \\
        \hline
        4                             & \doublecell{30}{3}  & \doublecell{$-$}{6} & \doublecell{$-$}{5} & \doublecell{100}{2} & 0              \\
        \hline
        5                             & \doublecell{5}{8}   & \doublecell{$-$}{6} & \doublecell{135}{2} & \doublecell{$-$}{5} & 0              \\
        \hline
        Потребность                   & 0                   & 0                   & 0                   & 0                   & \diagbox{0}{0} \\
        \hline
    \end{tabular}
\end{table}

Мы получили опорное решение, которое можно улучшить методом потенциалов.

\subsubsection{Метод потенциалов}\label{05-lab-potentials}

Я буду использовать опорное решение, полученное \hyperref[05-lab-nw-corner]{методом северо-западного угла}.

Опорное решение: $X_0 = \begin{pmatrix}
        170 & 0   & 0  & 0   \\
        25  & 40  & 0  & 0   \\
        0   & 115 & 0  & 0   \\
        0   & 35  & 95 & 0   \\
        0   & 0   & 40 & 100
    \end{pmatrix}$, значение целевой функции
\[f(X_0) = 170 \cdot 7 + 25 \cdot 7 + 40 \cdot 3 + 115 \cdot 4 + 35 \cdot 6 + 95 \cdot 5 + 40 \cdot 2 + 100 \cdot 5 = 3210\]

Элемент (клетка) называется \textbf{базисным}, если в нём написана цифра, а не прочерк. Элемент (клетка) называется \textbf{небазисным}, если в нём написан прочерк.

Для каждой клетки введём переменные $u_i$ и $v_j$ --- потенциалы складов и магазинов соответственно.

\textbf{Критерий оптимальности решения:} $\Delta_{ij} = u_i + v_j - c_{ij} \leq 0$, причём для базисного элемента $\Delta_{ij} = u_i + v_j - c_{ij} = 0$.

Сначала нам нужно расставить потенциалы. Пусть $u_1 = 0$, тогда $v_1 = 7$.

\begin{table}[H]
    \centering
    \begin{tabular}{|c|c|c|c|c|}
        \hline
        \backslashbox{$u_i$}{$v_j$} & 7                   &                     &                     &                     \\
        \hline
        0                           & \doublecell{170}{7} & \doublecell{$-$}{4} & \doublecell{$-$}{8} & \doublecell{$-$}{8} \\
        \hline
                                    & \doublecell{25}{7}  & \doublecell{40}{3}  & \doublecell{$-$}{7} & \doublecell{$-$}{8} \\
        \hline
                                    & \doublecell{$-$}{5} & \doublecell{115}{4} & \doublecell{$-$}{8} & \doublecell{$-$}{8} \\
        \hline
                                    & \doublecell{$-$}{3} & \doublecell{35}{6}  & \doublecell{95}{5}  & \doublecell{$-$}{2} \\
        \hline
                                    & \doublecell{$-$}{8} & \doublecell{$-$}{6} & \doublecell{40}{2}  & \doublecell{100}{5} \\
        \hline
    \end{tabular}
\end{table}

Базисных элементов в первой строке не осталось, переходим к первому столбцу.

$v_1 = 7$, значит для второй строки должно выполняться $u_2 = 0$.

\begin{table}[H]
    \centering
    \begin{tabular}{|c|c|c|c|c|}
        \hline
        \backslashbox{$u_i$}{$v_j$} & 7                   &                     &                     &                     \\
        \hline
        0                           & \doublecell{170}{7} & \doublecell{$-$}{4} & \doublecell{$-$}{8} & \doublecell{$-$}{8} \\
        \hline
        0                           & \doublecell{25}{7}  & \doublecell{40}{3}  & \doublecell{$-$}{7} & \doublecell{$-$}{8} \\
        \hline
                                    & \doublecell{$-$}{5} & \doublecell{115}{4} & \doublecell{$-$}{8} & \doublecell{$-$}{8} \\
        \hline
                                    & \doublecell{$-$}{3} & \doublecell{35}{6}  & \doublecell{95}{5}  & \doublecell{$-$}{2} \\
        \hline
                                    & \doublecell{$-$}{8} & \doublecell{$-$}{6} & \doublecell{40}{2}  & \doublecell{100}{5} \\
        \hline
    \end{tabular}
\end{table}

Базисных элементов в первом столбце не осталось, переходим ко второй строке.

$u_2 = 0$, значит для второго столбца должно выполняться $v_2 = 3$.

\begin{table}[H]
    \centering
    \begin{tabular}{|c|c|c|c|c|}
        \hline
        \backslashbox{$u_i$}{$v_j$} & 7                   & 3                   &                     &                     \\
        \hline
        0                           & \doublecell{170}{7} & \doublecell{$-$}{4} & \doublecell{$-$}{8} & \doublecell{$-$}{8} \\
        \hline
        0                           & \doublecell{25}{7}  & \doublecell{40}{3}  & \doublecell{$-$}{7} & \doublecell{$-$}{8} \\
        \hline
                                    & \doublecell{$-$}{5} & \doublecell{115}{4} & \doublecell{$-$}{8} & \doublecell{$-$}{8} \\
        \hline
                                    & \doublecell{$-$}{3} & \doublecell{35}{6}  & \doublecell{95}{5}  & \doublecell{$-$}{2} \\
        \hline
                                    & \doublecell{$-$}{8} & \doublecell{$-$}{6} & \doublecell{40}{2}  & \doublecell{100}{5} \\
        \hline
    \end{tabular}
\end{table}

Базисных элементов во второй строке не осталось, переходим ко второму столбцу. И так далее, пока потенциалы не будут расставлены.

Конечная таблица с потенциалами:

\begin{table}[H]
    \centering
    \begin{tabular}{|c|c|c|c|c|}
        \hline
        \backslashbox{$u_i$}{$v_j$} & 7                   & 3                   & 2                   & 5                   \\
        \hline
        0                           & \doublecell{170}{7} & \doublecell{$-$}{4} & \doublecell{$-$}{8} & \doublecell{$-$}{8} \\
        \hline
        0                           & \doublecell{25}{7}  & \doublecell{40}{3}  & \doublecell{$-$}{7} & \doublecell{$-$}{8} \\
        \hline
        1                           & \doublecell{$-$}{5} & \doublecell{115}{4} & \doublecell{$-$}{8} & \doublecell{$-$}{8} \\
        \hline
        3                           & \doublecell{$-$}{3} & \doublecell{35}{6}  & \doublecell{95}{5}  & \doublecell{$-$}{2} \\
        \hline
        0                           & \doublecell{$-$}{8} & \doublecell{$-$}{6} & \doublecell{40}{2}  & \doublecell{100}{5} \\
        \hline
    \end{tabular}
\end{table}

Для базисных элементов критерий оптимальности выполняется. Теперь рассматриваем небазисные элементы.

Например, для клетки $(3, 1)$ критерий не выполняется: $\Delta_{31} = u_3 + v_1 - c_{31} = 1 + 7 - 5 = 3 \nleq 0$.

\textbf{Алгоритм перехода к более оптимальному плану:}

\begin{enumerate}
    \item Выбираем небазисный элемент, для которого критерий оптимальности не выполняется. Помечаем его знаком $\oplus$.
    \item Строим цикл из четырёх клеток, включая помеченную клетку: 2 клетки в одном столбце и 2 клетки в одной строке, 3 базисных клетки и одна уже нами помеченная небазисная.
          Помечаем клетки, чередуя знаки $\oplus$ и $\ominus$.
    \item Находим $\theta = \min\{x_{ij} \mid \text{клетка } (i, j) \text{ помечена } \ominus\}$.
    \item Считаем новое значение $f_{\text{нов}} = f_{\text{стар}} + \theta * \Delta_{ij}$, где $ij$ - та клетка, для которой не выполняется критерий оптимальности.
    \item В новом оптимальном плане к клеткам, помеченным $\oplus$, прибавляем $\theta$, а из клеткок, помеченных $\ominus$, вычитается $\theta$.
    \item \textbf{Одну} клетку, в которых значение стало равно нулю, делаем небазисной.
\end{enumerate}

Итак, выберем клетку $(3, 1)$ и пометим её $\oplus$. Строим цикл: $(3, 1) \to (2, 1) \to (2, 2) \to (3, 2) \to (3, 1)$, помечая их, чередуя $\oplus$ и $\ominus$.

\begin{table}[H]
    \centering
    \begin{tabular}{|c|c|c|c|c|}
        \hline
        \backslashbox{$u_i$}{$v_j$} & 7                                  & 3                                   & 2                   & 5                   \\
        \hline
        0                           & \doublecell{170}{7}                & \doublecell{$-$}{4}                 & \doublecell{$-$}{8} & \doublecell{$-$}{8} \\
        \hline
        0                           & \othermarkedcell{25}{7}{$\ominus$} & \othermarkedcell{40}{3}{$\oplus$}   & \doublecell{$-$}{7} & \doublecell{$-$}{8} \\
        \hline
        1                           & \firstmarkedcell{5}                & \othermarkedcell{115}{4}{$\ominus$} & \doublecell{$-$}{8} & \doublecell{$-$}{8} \\
        \hline
        3                           & \doublecell{$\ominus$}{3}          & \doublecell{35}{6}                  & \doublecell{95}{5}  & \doublecell{$-$}{2} \\
        \hline
        0                           & \doublecell{$-$}{8}                & \doublecell{$-$}{6}                 & \doublecell{40}{2}  & \doublecell{100}{5} \\
        \hline
    \end{tabular}
\end{table}

При этом $\theta_1 = \min(x_{21}, x_{ 32}) = 25$

Пересчитываем целевую функцию: $ f(X_1) = f(X_0) - \theta_1 \cdot \Delta_{31} = 3210 - 25 \cdot 3 = 3135$

Новое опорное решение:

\begin{table}[H]
    \centering
    \begin{tabular}{|c|c|c|c|c|}
        \hline
        \backslashbox{$u_i$}{$v_j$} &                           &                     &                     &                     \\
        \hline
                                    & \doublecell{170}{7}       & \doublecell{$-$}{4} & \doublecell{$-$}{8} & \doublecell{$-$}{8} \\
        \hline
                                    & \doublecell{$0 \to -$}{7} & \doublecell{65}{3}  & \doublecell{$-$}{7} & \doublecell{$-$}{8} \\
        \hline
                                    & \doublecell{25}{5}        & \doublecell{90}{4}  & \doublecell{$-$}{8} & \doublecell{$-$}{8} \\
        \hline
                                    & \doublecell{$-$}{3}       & \doublecell{35}{6}  & \doublecell{95}{5}  & \doublecell{$-$}{2} \\
        \hline
                                    & \doublecell{$-$}{8}       & \doublecell{$-$}{6} & \doublecell{40}{2}  & \doublecell{100}{5} \\
        \hline
    \end{tabular}
\end{table}

Потенциалы пересчитываются тем же образом:

\begin{table}[H]
    \centering
    \begin{tabular}{|c|c|c|c|c|}
        \hline
        \backslashbox{$u_i$}{$v_j$} & 7                   & 6                   & 5                   & 8                   \\
        \hline
        0                           & \doublecell{170}{7} & \doublecell{$-$}{4} & \doublecell{$-$}{8} & \doublecell{$-$}{8} \\
        \hline
        -3                          & \doublecell{$-$}{7} & \doublecell{65}{3}  & \doublecell{$-$}{7} & \doublecell{$-$}{8} \\
        \hline
        -2                          & \doublecell{25}{5}  & \doublecell{90}{4}  & \doublecell{$-$}{8} & \doublecell{$-$}{8} \\
        \hline
        0                           & \doublecell{$-$}{3} & \doublecell{35}{6}  & \doublecell{95}{5}  & \doublecell{$-$}{2} \\
        \hline
        -3                          & \doublecell{$-$}{8} & \doublecell{$-$}{6} & \doublecell{40}{2}  & \doublecell{100}{5} \\
        \hline
    \end{tabular}
\end{table}

Теперь для клетки $(1, 2)$ критерий не выполняется: $\Delta_{12} = u_1 + v_2 - c_{12} = 0 + 6 - 4 = 2 \nleq 0$.

Построим цикл:

\begin{table}[H]
    \centering
    \begin{tabular}{|c|c|c|c|c|}
        \hline
        \backslashbox{$u_i$}{$v_j$} & 7                                   & 6                                  & 5                   & 8                   \\
        \hline
        0                           & \othermarkedcell{170}{7}{$\ominus$} & \firstmarkedcell{4}                & \doublecell{$-$}{8} & \doublecell{$-$}{8} \\
        \hline
        -3                          & \doublecell{$-$}{7}                 & \doublecell{65}{3}                 & \doublecell{$-$}{7} & \doublecell{$-$}{8} \\
        \hline
        -2                          & \othermarkedcell{25}{5}{$\oplus$}   & \othermarkedcell{90}{4}{$\ominus$} & \doublecell{$-$}{8} & \doublecell{$-$}{8} \\
        \hline
        0                           & \doublecell{$-$}{3}                 & \doublecell{35}{6}                 & \doublecell{95}{5}  & \doublecell{$-$}{2} \\
        \hline
        -3                          & \doublecell{$-$}{8}                 & \doublecell{$-$}{6}                & \doublecell{40}{2}  & \doublecell{100}{5} \\
        \hline
    \end{tabular}
\end{table}

$\theta_2 = \min(x_{11}, x_{ 32}) = 90,\ f(X_2) = f(X_1) - \theta_2 \cdot \Delta_{12} = 3135 - 90 \cdot 2 = 2955$

Новое опорное решение:

\begin{table}[H]
    \centering
    \begin{tabular}{|c|c|c|c|c|}
        \hline
        \backslashbox{$u_i$}{$v_j$} &                     &                     &                     &                     \\
        \hline
                                    & \doublecell{80}{7}  & \doublecell{90}{4}  & \doublecell{$-$}{8} & \doublecell{$-$}{8} \\
        \hline
                                    & \doublecell{$-$}{7} & \doublecell{65}{3}  & \doublecell{$-$}{7} & \doublecell{$-$}{8} \\
        \hline
                                    & \doublecell{115}{5} & \doublecell{$-$}{4} & \doublecell{$-$}{8} & \doublecell{$-$}{8} \\
        \hline
                                    & \doublecell{$-$}{3} & \doublecell{35}{6}  & \doublecell{95}{5}  & \doublecell{$-$}{2} \\
        \hline
                                    & \doublecell{$-$}{8} & \doublecell{$-$}{6} & \doublecell{40}{2}  & \doublecell{100}{5} \\
        \hline
    \end{tabular}
\end{table}

И так далее. После нескольких итераций (семи штук в данном случае) получаем оптимальное решение:

\begin{table}[H]
    \centering
    \begin{tabular}{|c|c|c|c|c|}
        \hline
        \backslashbox{$u_i$}{$v_j$} & 7                   & 4                   & 3                   & 6                   \\
        \hline
        0                           & \doublecell{45}{7}  & \doublecell{125}{4} & \doublecell{$-$}{8} & \doublecell{$-$}{8} \\
        \hline
        -1                          & \doublecell{$-$}{7} & \doublecell{65}{3}  & \doublecell{$-$}{7} & \doublecell{$-$}{8} \\
        \hline
        -2                          & \doublecell{115}{5} & \doublecell{$-$}{4} & \doublecell{$-$}{8} & \doublecell{$-$}{8} \\
        \hline
        -4                          & \doublecell{35}{3}  & \doublecell{$-$}{6} & \doublecell{$-$}{5} & \doublecell{95}{2}  \\
        \hline
        -1                          & \doublecell{$-$}{8} & \doublecell{$-$}{6} & \doublecell{135}{2} & \doublecell{5}{5}   \\
        \hline
    \end{tabular}
\end{table}

Здесь для всех клеток выполняется критерий оптимальности.

Таким образом, оптимальное решение $X^* = X_7 = \begin{pmatrix}
        45  & 125 & 0   & 0  \\
        0   & 65  & 0   & 0  \\
        115 & 0   & 0   & 0  \\
        35  & 0   & 0   & 95 \\
        0   & 0   & 135 & 5
    \end{pmatrix}$, при этом $f_{min} = f(X_7) = 2175$.

\subsubsection{Программная реализация}

\begin{lstlisting}[language=Python]
from cvxopt.modeling import variable, op
import time
start = time.time()
x = variable(25, 'x')
c= [7, 4, 8, 8, 7, 3, 7, 8, 5, 4, 8, 8, 3, 6, 5, 2, 8, 6, 2, 5]
z=sum(c[i]*x[i] for i in range(20))
mass = []
mass += [sum(x[i] for i in range(4)) == 170]
mass += [sum(x[i] for i in range(4, 8)) == 65]
mass += [sum(x[i] for i in range(8, 12)) == 115]
mass += [sum(x[i] for i in range(12, 16)) == 130]
mass += [sum(x[i] for i in range(16, 20)) == 140]
mass += [sum(x[i] for i in range(0, 20, 4)) == 195]
mass += [sum(x[i] for i in range(1, 20, 4)) == 190]
mass += [sum(x[i] for i in range(2, 20, 4)) == 135]
mass += [sum(x[i] for i in range(3, 20, 4)) == 100]
mass += [x >= 0]    

problem =op(z, mass)
problem.solve(solver='glpk')  
print("Результат Xopt:")
for i, value in enumerate(x.value, start=1):
    print(f"{int(value):3}", end=' ')
    if i % 4 == 0:
        print()
print(f"Стоимость доставки: {problem.objective.value()[0]}")
stop = time.time()
print (f"Время:{stop - start}")
\end{lstlisting}

Вывод:
\begin{lstlisting}[language=text]
Xopt result:
 45 125   0   0 
  0  65   0   0 
115   0   0   0 
 35   0   0  95 
  0   0 135   5 
  0   0   0   0 
Transport cost: 2175.0
Time: 0.012
\end{lstlisting}

\textbf{Ответ:} Оптимальное решение --- $X^* = \begin{pmatrix}
        45  & 125 & 0   & 0  \\
        0   & 65  & 0   & 0  \\
        115 & 0   & 0   & 0  \\
        35  & 0   & 0   & 95 \\
        0   & 0   & 135 & 5
    \end{pmatrix}$,

минимум целевой функции --- $f_{min} = 2175$. \label{05-lab-answer}

