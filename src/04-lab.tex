\section{Задание №4 (Вариант 3)}\label{04-lab}

\subsection{Условие}\label{04-lab-condition}
Для следующего текстового условия составить математическую модель для прямой и двойственной задачи. Получить решение прямой и двойственной задач
симплекс-методом. Дать экономическую интерпретацию двойственных задачи двойственных оценок.

Для производства четырех видов изделий $(A, B, C)$ предприятие использует три вида сырья: металл, пластмассу, резину. 
Запасы сырья, технологические коэффициенты (расход каждого вида сырья на производство единицы каждого изделия)
представлены в Таблице \ref{04-lab-01-table}. В ней же указана прибыль от реализации одного изделия каждого вида. 
Требуется составить такой план выпуска указанных изделий, чтобы обеспечить максимальную прибыль.

\begin{table}[H]
    \centering
    \begin{tabular}{|>{\columncolor{lightgray}}c|c|c|c|c|c|}
    \hline
        \rowcolor{lightgray} Сырьё & A & B & C & D & Запасы \\ \hline
        Металл & 9 & 7 & 8 & 4 & 1500 \\ \hline
        Пластмасса & 6 & 1 & 4 & 2 & 1000 \\ \hline
        Резина & 3 & 1 & 2 & 0 & 700 \\ \hline
        \textbf{Прибыль} (руб) & 1 & 2 & 4 & 5 & ~ \\ \hline
    \end{tabular}
    \caption{}
    \label{04-lab-01-table}
\end{table}

\subsection{Решение}\label{04-lab-solution}

\subsubsection{Математические модели}

Прямая задача:

\begin{align*}
    &F(x_{ 1}, x_{ 2}, x_{ 3} , x_{ 4}) = x_{ 1} + 2x_{ 2} + 4x_{ 3} + 5x_4 \to \max\\
    &\begin{cases}
        9x_1 + 7x_2 + 8x_3 + 4x_4 \leq 1500\\
        6x_1 + x_2 + 4x_3 + 2x_4 \leq 1000\\
        3x_1 + x_2 + 2x_3 \leq 700
    \end{cases}\\
    &x_i \geq 0, \ \forall i\in\overline{1,4}
\end{align*}

Обратная задача:

\begin{align*}
    &F^{ *}(y_{ 1}, y_{ 2}, y_{ 3}) = 1500y_{ 1} + 1000y_{ 2} + 700y_{ 3} \to \min\\
    &\begin{cases}
        9y_1 + 6y_2 + 3y_3 \geq 1\\
        7y_1 + y_2 + y_3 \geq 2\\
        8y_1 + 4y_2 + 2y_3 \geq 4\\
        4y_1 + 2y_2 \geq 5
    \end{cases}\\
    &y_i \geq 0, \ \forall i\in\overline{1,4}
\end{align*}

\subsubsection{Решение прямой задачи симплекс-методом}

Приведём прямую задачу к каноническому виду:

\begin{align*}
    &F - x_{ 1} - 2x_{ 2} - 4x_{ 3} - 5x_4 = 0\\
    &\begin{cases}
        9x_1 + 7x_2 + 8x_3 + 4x_4 + x_5 = 1500\\
        6x_1 + x_2 + 4x_3 + 2x_4 + x_6 = 1000\\
        3x_1 + x_2 + 2x_3 + x_7 = 700
    \end{cases}\\
\end{align*}

И решим её с помощью симплекс-таблиц:

\begin{table}[H]
    \centering
    \begin{tabular}{|c|c|c|c|c|>{\columncolor{mycolumncolor}}c|c|c|c|c|c|}
    \hline
         & Базис & $x_1$ & $x_2$ & $x_3$ & $x_4$ & $x_5$ & $x_6$ & $x_7$ & $b_i$ & $\frac{b_i}{\text{разрешающий столбец}}$ \\ \hline
         \myrowcolor
         & $x_5$ & 9 & 7 & 8 & \mycellcolor4 & 1 & 0 & 0 & 1500 & 375 \\ \hline
         $\mathbf{\cdot 2}$ & $x_6$ & 6 & 1 & 4 & 2 & 0 & 1 & 0 & 1000 & 500 \\ \hline
         & $x_7$ & 3 & 1 & 2 & 0 & 0 & 0 & 1 & 700 & +\infty \\ \hline
         $\mathbf{\cdot 4}$ & F & -1 & -2 & -4 & -5 & 0 & 0 & 0 &  &  \\ \hline
    \end{tabular}
    \caption{}
    \label{04-lab-02-table}
\end{table}

\begin{table}[H]
    \centering
    \begin{tabular}{|c|c|c|c|c|>{\columncolor{mycolumncolor}}c|c|c|c|c|c|}
    \hline
        & Базис & $x_1$ & $x_2$ & $x_3$ & $x_4$ & $x_5$ & $x_6$ & $x_7$ & $b_i$ & $\frac{b_i}{\text{разрешающий столбец}}$ \\ \hline
        \myrowcolor
         & $x_5$ & 9 & 7 & 8 & \mycellcolor4 & 1 & 0 & 0 & 1500 & 375 \\ \hline
         & $x_6$ & 12 & 2 & 8 & 4 & 0 & 2 & 0 & 2000 & 500 \\ \hline
         & $x_7$ & 3 & 1 & 2 & 0 & 0 & 0 & 1 & 700 & +\infty \\ \hline
         & 4F & -4 & -8 & -16 & -20 & 0 & 0 & 0 & 0 &  \\ \hline
    \end{tabular}
    \caption{}
    \label{04-lab-03-table}
\end{table}

\begin{table}[H]
    \centering
    \begin{tabular}{|c|c|c|c|c|c|c|c|c|c|c|}
    \hline
         & Базис & $x_1$ & $x_2$ & $x_3$ & $x_4$ & $x_5$ & $x_6$ & $x_7$ & $b_i$ & $\frac{b_i}{\text{разрешающий столбец}}$ \\ \hline
         & $x_4$ & 9 & 7 & 8 & 4 & 1 & 0 & 0 & 1500 &  \\ \hline
         & $x_6$ & 3 & -5 & 0 & 0 & -1 & 2 & 0 & 500 &  \\ \hline
         & $x_7$ & 3 & 1 & 2 & 0 & 0 & 0 & 1 & 700 &  \\ \hline
         & 4F & 41 & 27 & 24 & 0 & 5 & 0 & 0 & 7500 & \text{Все > 0} \\ \hline
    \end{tabular}
    \caption{}
    \label{04-lab-04-table}
\end{table}

Мы решили задачу оптимизации, перейдя в точку $\overrightarrow{x} = \left(0; 0; 0; 375; 0; 250; 700\right)$
А также получили выражение для максимума: $ 4F_{ \max} + 41x_1 + 27x_2 + 24x_3 + 5x_5 = 7500 $, откуда $ F_{ \max} = 1875 $ 

\subsubsection{Решение двойственной задачи по третьей теореме двойственности}

Смотрим в таблицу \ref{04-lab-04-table}: базисные элементы $ x_4 $, $ x_6 $ и $ x_7 $.

\begin{enumerate}
    \item Коэффициенты в функции перед $ x_4 $, $ x_6 $ и $ x_7 $ : $ \overrightarrow{c} = \begin{pmatrix}
        5 & 0 & 0
    \end{pmatrix} $
    \item Берём из таблицы \ref{04-lab-02-table} столбцы при $ x_4 $, $ x_6 $ и $ x_7 $ и получаем матрицу 
    $ 
    \begin{pmatrix}
        4 & 0 & 0 \\
        4 & 1 & 0\\
        0 & 0 & 1 
    \end{pmatrix} 
    $.\\
    \item Находим для неё обратную матрицу $ A^{ -1}_{ B} = \begin{pmatrix}
        \frac{ 1}{ 4}   & 0 & 0\\
        -1 & 1 & 0\\
        0 & 0 & 1
    \end{pmatrix} $
\end{enumerate}

По теореме находим $ \mathbf{y^*}  $:

\[ \mathbf{y^*}= \begin{pmatrix}
    y_1^* & y_2^* & y_3^*
\end{pmatrix}= \overrightarrow{c} \cdot A_B^{ -1} = \begin{pmatrix}
    5 & 0 & 0
\end{pmatrix} \cdot 
\begin{pmatrix}
    \frac{ 1}{ 4}   & 0 & 0\\
    -1 & 1 & 0\\
    0 & 0 & 1
\end{pmatrix} = \begin{pmatrix}
        \frac{ 5}{ 4} & 0 & 0
    \end{pmatrix}
;\]

Подставив в функцию $ F^{ *}(y_{ 1}, y_{ 2}, y_{ 3}) = 1500y_{ 1} + 1000y_{ 2} + 700y_{ 3} $, получаем то же, что получили предыдущим способом: $ F^{ *} = F_{ \min}^{ *} = 1875 $,
значение сошлось с решением прямой задачи.

\subsubsection{Экономическая интерпретация двойственной задачи и двойственных оценок}

Рассматривая математическую интерпретацию прямой и двойственных задач, и находя связь между экономической интерпретацией прямой задачи,
можно сделать вывод, что $ y_1, y_2, y_3 $ измеряются в деньгах. Далее, рассматривая выражение 
$ F^{ *}(y_{ 1}, y_{ 2}, y_{ 3}) = 1500y_{ 1} + 1000y_{ 2} + 700y_{ 3} \to \min$, можно сделать вывод, что
мы хотим уменьшить затраты на закупку необходимых ресурсов при условии прибыльности. 

Левая часть неравенств в двойственной задаче обозначает денежные затраты на производство изделий $ A, B, C, D $,
правая часть неравентсв - прибыль от продажи этих изделий. А функция - затраты на закупку материалов.
При этом мы хотим, чтобы наше производство было неубыльно.

То есть двойственная задача нам предлагает найти такие цены на ресурсы, при которых у нас минимальные затраты на закупку ресурсов, при
этом производство либо прибыльно, либо выходит в "ноль".